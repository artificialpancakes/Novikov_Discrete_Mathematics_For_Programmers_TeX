\vspace{5pt} \hrule
\begin{lstlisting}[caption={Алгоритм Флойда}, label=p285, escapechar=\%]
%\noindent\textbf{Вход:} матрица C[1..p,1..p] длин дуг.\\%
%\textbf{Выход:} матрица T[1..p,1..p] длин путей и матрица H[1..p,1..p] самих путей.%
void floyd_algorithm(int p, int C[p][p], T[p][p], H[p][p])
{
	for(int i = 0; i < p; ++i)
	{
		for(int j = 0; j < p; ++j)
		{
			T[i][j] = C[i][j]; /* инициализация */
			if(C[i][j] == INT_MAX)
				H[i][j] = 0; /* нет дуги из i в j */
			else
				H[i][j] = j; /* есть дуга из i в j */
		}
	}
	for(int i = 0; i < p; ++i)
	{
		for(int j = 0; j < p; ++j)
		{
			for(int k = 0; k < p; ++k)
			{
				if(i != j && T[j][i] != INT_MAX && i != k && T[i][k] != INT_MAX && (T[j][k] == INT_MAX || T[j][k] > (T[j][i] + T[i][k])))
				{
					H[j][k] = H[j][i]; /* запомнить новый путь */
					T[j][k] = T[j][i] + T[i][k]; /* и его длину */
				}
			}
		}
		for(int j = 0; j < p; ++j)
		{
			if(T[j][j] < 0)
				return; /* нет решения: вершина j входит в цикл отрицательной длины */
		}
	}
}
\end{lstlisting}
