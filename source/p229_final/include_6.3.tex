\vspace{5pt} \hrule
\begin{lstlisting}[caption={Упаковка по методу Лемпела - Зива}, label=p229, escapechar=\%]
%\noindent\textbf{Вход:} исходный текст, заданный массивом кодов символов f: \textbf{array} [1..n] \textbf{of char}.\\%
%\textbf{Выход:} сжатый текст, представленный последовательностью пар <p, q>, где p - номер слова в словаре, q - код дополняющей буквы.%
void pack(int n, char *f)
{
	D[0][0] = '\0';
	d = 0; /* начальное состояние словаря */
	int k = 0; /* номер текущей буквы в исходном тексте */
	while(k < n)
	{
		int p = FD(k, f); /* индекс найденного слова в словаре */
		int l = Length(p); /* l - длинна найденного слова в словаре */
		push(Res, p, f[k + l]);	/* код найденного слова и еще одна буквы добавляются в вектор */
		d = d + 1;
		strcpy(D[d], D[p]);
		D[d][l] = f[k + l]; /* пополнение словаря */
		k = k + l + 1; /* продвижение вперед по исходному тексту */
	}
}
\end{lstlisting}
