\vspace{5pt} \hrule

\begin{lstlisting}[caption={Вычисление номера кортежа в установленном порядке}, label=p134_calc_cortege_number, escapechar=`]
`\noindent\textbf{Вход:} вектор $x$ : \textbf{array} [1..$n$] \textbf{of string} идентификаторов переменных, вектор F : \textbf{array} [0..$2^n - 1$] \textbf{of} 0..1 значений функции при установленном порядке кортежей.\\`
`\textbf{Выход:} последовательность символов, образующих запись формулы СДНФ для заданной функции.`
vector* pdnf_construction(int n, char **x, int *F) {
  /* последовательность символов,
  образующих запись формулы СДНФ для заданной функции. */
  vector *out = vector_create();
  bool f = false; /* признак присутствия левого операнда дизъюнкции */
  for (int i = 0; i < pow(2, n); i++) {
    if (F[i] == 1) {
      if (f) {
        /* добавление в формулу знака дизъюнкции */
        vector_push_back(out, "`$\vee$`");
      } else {
        f = true; /* это первое слагаемое в дизъюнкции */
      }
      bool g = false; /* признак присутствия левого операнда конъюнкции */
      for (int j = 1; j <= n; j++) {
        if (g) {
          /* добавление в формулу знака конъюнкции */
          vector_push_back(out, "`$\wedge$`");
        } else {
          g = true; /* это первый сомdножитель в конъюнкции */
        }
        int v = (i >> (j - 1)) % 2;
        if (v == 0) {
          /* добавление в формулу знака отрицания */
          vector_push_back(out, "`$\neg$`");
        }
        /* добавление в формулу идентификатора переменной */
        vector_push_back(out, x[j - 1]);
      }
    }
  }
  return out;
}
\end{lstlisting}
