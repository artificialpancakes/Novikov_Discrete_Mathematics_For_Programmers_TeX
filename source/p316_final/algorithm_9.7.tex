\documentclass{article}

\usepackage[T2A]{fontenc}
\usepackage[utf8]{inputenc}
\usepackage[english, russian]{babel}
\usepackage[left = 2cm]{geometry}
\usepackage{listings}
\usepackage{amssymb}
\usepackage{caption}

\addto\captionsrussian{
  \renewcommand{\lstlistingname}{Алгоритм}
}
\DeclareCaptionFormat{listing}{\par\vskip1pt#1#2#3}
\captionsetup[lstlisting]{format=listing, singlelinecheck=false, labelfont=bf}

\lstset{
  language=C,
  extendedchars=\true,
  keepspaces=true,
  breaklines=true,
  frame=lines,
  aboveskip=-5pt,
  tabsize=1,
  morekeywords={Вход:, Выход:, true, false, bool, NULL}
}

\begin{document}

\vspace{5pt} \hrule
\begin{lstlisting}[caption={Бинарный поиск}, label=p316_bin_search, escapechar=\%]
%\noindent\textbf{Вход:} упорядоченный массив A: \textbf{array} [1..n] \textbf{of record} k: key; i: info \textbf{end record}; ключ a: key.\\%
%\textbf{Выход:} индекс записи с искомым ключом a в массиве A или 0, если записи с таким ключом нет.%
int bin_search(int n, record A[n], int a)
{
	int b = 0; /* начальный индекс части массива для поиска */
	int e = n - 1; /* конечный индекс части массива для поиска */
	while(b <= e)
	{
		int c = (b + e)/2; /* индекс проверяемого элемента %(округленный до целого)% */
		if(A[c].k > a)
			e = c - 1; /* продолжаем поиск в первой половине */
		else if(A[c].k < a)
			b = c + 1; /* продолжаем поиск во второй половине */
		else
			return c + 1; /* нашли искомый ключ */
	}
	return 0; /* искомого ключа нет в массиве */
}
\end{lstlisting}

\end{document}
