\documentclass{article}

\usepackage[T2A]{fontenc}
\usepackage[utf8]{inputenc}
\usepackage[english, russian]{babel}
\usepackage[left = 2cm]{geometry}
\usepackage{listings}
\usepackage{amssymb}
\usepackage{caption}

\addto\captionsrussian{
  \renewcommand{\lstlistingname}{Алгоритм}
}
\DeclareCaptionFormat{listing}{\par\vskip1pt#1#2#3}
\captionsetup[lstlisting]{format=listing, singlelinecheck=false, labelfont=bf}

\lstset{
  language=C,
  extendedchars=\true,
  keepspaces=true,
  breaklines=true,
  frame=lines,
  aboveskip=-5pt,
  tabsize=1,
  morekeywords={Вход:, Выход:, true, false, bool, NULL, node}
}

\begin{document}

\vspace{5pt} \hrule
\begin{lstlisting}[caption={Поиск узла в дереве сортировки}, label=p316_search_node, escapechar=\%]
%\noindent\textbf{Вход:} дерево сортировки T, заданное указателем на корень; ключ a: key.\\%
%\textbf{Выход:} указатель p на найденный узел или \textbf{NULL}, если в дереве нет такого ключа.%
node *search_node(node *T, int a)
{
	node *p = T; /* указатель на проверяемый узел */
	while(p != NULL)
	{
		if(a < p->i)
			p = p->l; /* продолжаем поиск слева */
		else if(a > p->i)
			p = p->r; /* продолжаем поиск справа */
		else
			return p; /* нашли узел */
	}
	return NULL; /* искомого ключа нет в дереве */
}
\end{lstlisting}

\end{document}
