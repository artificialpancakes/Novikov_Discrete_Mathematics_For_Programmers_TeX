\documentclass{article}

\usepackage[T2A]{fontenc}
\usepackage[utf8]{inputenc}
\usepackage[english, russian]{babel}
\usepackage[left = 2cm]{geometry}
\usepackage{listings}
\usepackage{amssymb}
\usepackage{caption}

\addto\captionsrussian{
  \renewcommand{\lstlistingname}{Алгоритм}
}
\DeclareCaptionFormat{listing}{\par\vskip1pt#1#2#3}
\captionsetup[lstlisting]{format=listing, singlelinecheck=false, labelfont=bf}

\lstset{
  language=C,
  extendedchars=\true,
  keepspaces=true,
  breaklines=true,
  frame=lines,
  aboveskip=-5pt,
  tabsize=1,
  morekeywords={Вход:, Выход:, true, false, bool, NULL}
}

\begin{document}

\vspace{5pt} \hrule
\begin{lstlisting}[caption={Генерация всех подмножеств n-элементного множества}, label=p_39, escapechar=\%]
%\noindent\textbf{Вход:} $n \geqslant 0$ - мощность множества.\\%
%\textbf{Выход:} последовательность кодов подмножеств i.%
int *create_subsets(int n)
{
	int s = pow(2, n), m[s];
	for(int i = 0; i < s; ++i)
		m[i] = i; /* код очередного подмножества */
	return m;
}
\end{lstlisting}

\end{document}
