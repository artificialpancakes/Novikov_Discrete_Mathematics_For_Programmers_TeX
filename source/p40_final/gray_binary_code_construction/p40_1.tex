\documentclass{article}

\usepackage[T2A]{fontenc}
\usepackage[utf8]{inputenc}
\usepackage[russian, english]{babel}
\usepackage[left = 2cm]{geometry}
\usepackage{listings}
\usepackage{amssymb}
\usepackage{caption}

\renewcommand{\lstlistingname}{\textbf{Алгоритм}}
\DeclareCaptionFormat{listing}{\par\vskip1pt#1#2#3}
\captionsetup[lstlisting]{format=listing, singlelinecheck=false, labelfont=bf}

\lstset{
  language=C,
  extendedchars=\true,
  keepspaces=true,
  breaklines=true,
  frame=lines,
  aboveskip=-5pt,
  tabsize=1,
  morekeywords={Вход:, Выход:, true, false, bool, NULL, vector, bitset}
}

\begin{document}

\vspace{5pt} \hrule
\begin{lstlisting}[caption={построение бинарного кода Грея}, label=gray_binary_code, escapechar=\%]
%\noindent\textbf{Вход:} n $\geqslant$ 0 — мощность множества.\\%
%\textbf{Выход:} последовательность кодов подмножеств В.%
vector* gray_binary_code(int n) {
  vector *out = vector_create(); /* последовательность кодов подмножеств B */
  bitset *B = bitset_create(n); /* битовая шкала для представления подмножеств */
  for (int i = 0; i < n; i++) {
    bitset_set(B, i, 0); /* инициализация */
  }
  vector_push_back(out, B); /* пустое множество */

  for (int i = 1; i < pow(2, n); i++) {
    /* определение номера элемента, подлежащего добавлению или удалению */
    int p = Q(i);
    /* добавление или удаление элемента */
    bitset_set(B, p, 1 - bitset_get(B, p));
    vector_push_back(out, B);
  }

  return out;
}
\end{lstlisting}

\end{document}
