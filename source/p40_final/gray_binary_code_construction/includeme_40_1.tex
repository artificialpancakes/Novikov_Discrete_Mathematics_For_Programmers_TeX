\vspace{5pt} \hrule
\begin{lstlisting}[caption={Построение бинарного кода Грея}, label=p40_gray_binary_code, escapechar=\%]
%\noindent\textbf{Вход:} n $\geqslant$ 0 — мощность множества.\\%
%\textbf{Выход:} последовательность кодов подмножеств В.%
vector* gray_binary_code(int n) {
  vector *out = vector_create(); /* последовательность кодов подмножеств B */
  bitset *B = bitset_create(n); /* битовая шкала для представления подмножеств */
  for (int i = 0; i < n; i++) {
    bitset_set(B, i, 0); /* инициализация */
  }
  vector_push_back(out, B); /* пустое множество */

  for (int i = 1; i < pow(2, n); i++) {
    /* определение номера элемента, подлежащего добавлению или удалению */
    int p = Q(i);
    /* добавление или удаление элемента */
    bitset_set(B, p, 1 - bitset_get(B, p));
    vector_push_back(out, B);
  }

  return out;
}
\end{lstlisting}
