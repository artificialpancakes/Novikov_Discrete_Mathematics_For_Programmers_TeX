\documentclass{article}

\usepackage[T2A]{fontenc}
\usepackage[utf8]{inputenc}
\usepackage[russian, english]{babel}
\usepackage[left = 2cm]{geometry}
\usepackage{listings}
\usepackage{amssymb}
\usepackage{caption}

\renewcommand{\lstlistingname}{\textbf{Алгоритм}}
\DeclareCaptionFormat{listing}{\par\vskip1pt#1#2#3}
\captionsetup[lstlisting]{format=listing, singlelinecheck=false, labelfont=bf}

\lstset{
  language=C,
  extendedchars=\true,
  keepspaces=true,
  breaklines=true,
  frame=lines,
  aboveskip=-5pt,
  tabsize=1,
  morekeywords={Вход:, Выход:, true, false, bool, NULL, vector, bitset}
}

\begin{document}

\vspace{5pt} \hrule
\begin{lstlisting}[caption={Функция Q определения номера изменяемого разряда}, label=p40_Q, escapechar=`]
`\noindent\textbf{Вход:} i — номер подмножества.\\`
`\textbf{Выход:} номер изменяемого разряда.`
int Q(int i) {
  int q = 1; int j = 1;
  while (j % 2 == 0) {
    j = j / 2; q++;
  }
  return q;
}
\end{lstlisting}

\end{document}
