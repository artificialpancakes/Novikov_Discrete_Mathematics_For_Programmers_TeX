\documentclass{article}

\usepackage[T2A]{fontenc}
\usepackage[utf8]{inputenc}
\usepackage[english, russian]{babel}
\usepackage[left = 2cm]{geometry}
\usepackage{listings}
\usepackage{amssymb}
\usepackage{caption}

\lstset{
  language=C,
  extendedchars=\true,
  keepspaces=true,
  breaklines=true,
  aboveskip=-5pt,
  tabsize=1,
  morekeywords={Вход:, Выход:, true, false, bool, NULL, node}
}

\begin{document}

\begin{lstlisting}[escapechar=\%]
%2. Удаление узла - процедура Delete.%
%\noindent\textbf{Вход:} p1 - указатель на удаляемый узел; p2 - указатель на подцепляющий узел; p3 - указатель на подцепляемый узел; s - способ подцепления.%
%\textbf{Выход:} преобразованное дерево.%
void Delete(node *p1, node *p2, node *p3, int s)
{
	if(s == -1)
		p2->l = p3; /* подцепляем слева */
	if(s == +1)
		p2->r = p3; /* подцепляем справа */
	free(p1); /* удаляем узел */
}
\end{lstlisting}

\end{document}
