\vspace{5pt} \hrule
\begin{lstlisting}[caption={Алгоритм вычисления значения СДНФ}, label=p_137, escapechar=\%]
%\noindent\textbf{Вход:} массив, представляющий СДНФ: f: \textbf{array} [1..k,1..n] \textbf{of} 0..1; множество значений переменных x: \textbf{array} [1..n] \textbf{of} 0..1.\\%
%\textbf{Выход:} 0..1 - значение булевой функции.%
int calc_CDNF(int k, int n, int f[k][n], int x[n])
{
	for(int i = 0; i < k; ++i)
	{
		int flag = 0;
		for(int j = 0; j < n; ++j)
		{
			if(f[i][j] != x[j])
				flag = 1; /* %$x_j \neq\sigma_j\implies x_j^{\sigma_j} = 0\implies x_1^{\sigma_1} \land\ldots\land x_n^{\sigma_n} = 0$% */
		}
		if(flag) continue;
		return 1; /* %$x_1^{\sigma_1} \,\&\ldots\&\, x_n^{\sigma_n} = 1 \implies \bigvee_{(\sigma_1, \,\dots\,, \sigma_m)} x_1^{\sigma_1} \land\ldots\land x_n^{\sigma_n} = 1$% */
	}
	return 0; /* все слагаемые в дизъюнкции равны нулю */
}
\end{lstlisting}
