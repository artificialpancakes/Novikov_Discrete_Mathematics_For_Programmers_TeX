\vspace{5pt} \hrule
\begin{lstlisting}[caption={Алгоритм Дейкстры}, label=p286_dijkstra_1, escapechar=`]
`\noindent\textbf{Вход:} орграф $G(V,E)$, заданный матрицей длин дуг \\$c$ : array [1..$p$, 1..$p$] \textbf{of real}; $s$ и $t$ — вершины графа.\\`
`\textbf{Выход:} векторы $T$ : \textbf{array} [1..$p$] \textbf{of real}; и $H$ : \textbf{array} [1..$p$] of 0..$p$. Если вершина $v$ лежит на кратчайшем пути от $s$ к $t$, то $T[v]$ — длина кратчайшего пути от $s$ к $v$; $H[v]$ — вершина, непосредственно предшествующая $v$ на кратчайшем пути.`
result dijkstra_algorithm(int p, double *C, int s, int t) {
  result res;
  double *T = malloc(sizeof(double) * p);
  int *H = malloc(sizeof(int) * p);
  res.T = T; res.H = H;
  double X[p];
  for (int i = 0; i < p; i++) {
    T[i] = INFINITY; /* кратчайший путь неизвестен */
    X[i] = 0; /* все вершины не отмечены */
  }
  H[s] = -1; /* ничего не предшествует */
  T[s] = 0; /* кратчайший путь имеет длину 0, */
  X[s] = 1; /* и он известен */
  int v = s; /* текущая вершина */
  M: /* обновление пометок */
  for (int u = 0; u < p; u++) {
    double length = C[v + u * p];
    if (length == INFINITY) {
      continue;
    }
    if ((X[u] == 0) && (T[u] > (T[v] + length))) {
      /* найден более короткий путь из s в u через v */
      T[u] = T[v] + length;
      H[u] = v; /* запоминаем его */
    }
  }
  double m = INFINITY; v = -1;
  /* поиск конца кратчайшего пути */
  for (int u = 0; u < p; u++) {
    if (X[u] == 0 && T[u] < m) {
      /* вершина v заканчивает кратчайший путь из s */
      v = u; m = T[u];
    }
  }
  if (v == t) {
    /* нет пути из s в t */
    res.flag = 0;
    return res;
  }
  if (v == -1) {
    /* найден кратчайший путь из s в t */
    res.flag = -1;
    return res;
  }
  /* найден кратчайший путь из s в v */
  X[v] = 1;
  goto M;
}
\end{lstlisting}
