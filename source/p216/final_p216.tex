\vspace{5pt} \hrule

\begin{lstlisting}[caption={Построение кодирования, близкого к оптимальному}, label=p138_calc_function, escapechar=`]
`\noindent\textbf{Вход:} $double P[n]$ - массив вероятностей появления букв в сообщении, упорядоченный по возрастанию; $1 \geqP[1] \geq \ldots P[n] \geq 0$, $P[1] + /ldots + P[n] = 1$.\\`
`\textbf{Выход:} $bool C[n][L]$ - массив элементарных кодов.`
void Fano(double *P, int **C, int b, int e, int k, int n){
int *C_arr = (int*)C;
if(e > b){
		int m = Med(P, b, e); /* деление массива на две части */
		for(int i = b; i <= e; i++){
			if(i > m) 
				C_arr[i * n + k] = 1; /* во второй части добавляем 1 */
			else
				C_arr[i * n + k] = 0; /* в первой части добавляем 0* /
		}
		Fano(P, (int**)C, b, m, k + 1, n); /* обработка первой части */
		Fano(P, (int**)C, m + 1, e, k + 1, n); /* оработка второй части */
}
\end{lstlisting}
