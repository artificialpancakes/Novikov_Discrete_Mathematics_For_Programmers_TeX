\vspace{5pt} \hrule

\begin{lstlisting}[caption={Генерация $n$-элементарных подмножеств $m$-элементарного множества}, label=p138_calc_function, escapechar=`]
`\noindent\textbf{Вход:} $n$ - мощность подмножества, $m$ - мощность множества, $m>n>0$.`
`\textbf{Выход:} последовательность всех $n$-элементных подмножеств m-элементного множества в лексикографическом порядке.`
        int A[m + 1];
        for(int i = 1; i <= m; i++){
                A[i] = i; /* инициализация исходного множества */
        }
        if(m == n){
                for(int i = 1; i <= n; i++)
                        printf("%d ", A[i]); /* единственное подмножество*/
                printf("\n");
        }
        int p = n; /* p - номер первого изменяемого элемента */
        while(p >= 0){
                for(int i = 1; i <= n; i++)
                        printf("%d ", A[i]); /* очередное подмножество */
                printf("\n");
                if(A[n] == m)
                        p--; /* нельзя увеличить элемент */
                else
                        p = n; /* можно увеличить элемент */
                if(p >= 1){
                        for(int i = n; i >= p; i--){
                                A[i] = A[p] + i - p + 1; /*увеличение элементов*/
                        }
                }
        }
\end{lstlisting}
