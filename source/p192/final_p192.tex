\vspace{5pt} \hrule

\begin{lstlisting}[caption={Генерация $n$-элементарных подмножеств $m$-элементарного множества}, label=p138_calc_function, escapechar=`]
\noindent\textbf{Вход:} $n$ - мощность подмножества, $m$ - мощность множества, $m>n>0$.
\textbf{Выход:} последовательность всех $n$-элементных подмножеств m-элементного множества в лексикографическом порядке.
bool calc_func(tree_node *T, int n, int *x) {
  for (int i = 0; i < n; i++) {
    if (T->l == NULL && T->r == NULL) {
      return T->i; /* листовой узел — возвращаем значениe */
    }
    else {
      if (x[i]) {
        T = T->r; /* 1 — переход вправо */
      } else {
        T = T->l; /* 0 — переход влево */
      }
    }
  }
}
\end{lstlisting}
