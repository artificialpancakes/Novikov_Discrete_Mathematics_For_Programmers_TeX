\vspace{5pt} \hrule
\begin{lstlisting}[caption={Алгоритм Дейкстры}, label=p286_dijkstra_1, escapechar=`]
`\noindent\textbf{Вход:} орграф $G(V,E)$, заданный матрицей длин дуг \\$c$ : array [1..$p$, 1..$p$] \textbf{of real}; $s$ - исходная вершина графа.\\`
`\textbf{Выход:} вектор $T$ : \textbf{array} [1..$p$] \textbf{of real} длин кратчайших путей от $s$.`
double* dijkstra_algorithm(int p, double *C, int s) {
  double *T = malloc(sizeof(double) * p);
  int X[p];
  for (int i = 0; i < p; i++) {
    /* начальное приближение определяется матрицей */
    T[i] = C[s + i * p];
    X[i] = 0; /* все вершины не отмечены */
  }
  for (int i = 0; i < p; i++) {
    int v;
    /* поиск конца нового кратчайшего пути */
    double m = INFINITY;
    for (int u = 0; u < p; u++) {
      if ((X[u] == 0) && (T[u] < m)) {
        /* вершина v заканчивает новый кратчайший путь из s */
        v = u; m = T[u];
      }
    }
    for (int u = 0; u < p; u++) {
      double length = C[v + u * p];
      if (length == INFINITY) {
        continue;
      }
      /* пересчёт оценки длины пути из s в u через v */
      T[u] = fmin(T[u], T[v] + length);
    }
    X[v] = 1; /* найден кратчайший путь из s в v */
  }
  return T;
}
\end{lstlisting}
