\documentclass{article}

\usepackage[T2A]{fontenc}
\usepackage[utf8]{inputenc}
\usepackage[english, russian]{babel}
\usepackage[left = 2cm]{geometry}
\usepackage{listings}
\usepackage{amssymb}
\usepackage{caption}

\lstset{
  language=C,
  extendedchars=\true,
  keepspaces=true,
  breaklines=true,
  aboveskip=-5pt,
  tabsize=1,
  morekeywords={Вход:, Выход:, true, false, bool, NULL, node}
}

\begin{document}

\begin{lstlisting}[escapechar=\%]
%1. Поиск узла - функция Find.%
%\noindent\textbf{Вход:} дерево сортировки T, заданное указателем на корень; ключ a: key.%
%\textbf{Выход:} p - указатель на найденный узел или \textbf{NULL}, если в дереве нет такого ключа; q - указатель на отца узла p; s - способ подцепления узла q к узлу p (s = -1, если p слева от q; s = +1, если p справа от q; s = 0, если p - корень).%
void Find(node *T, int a, node **p, node **q, int s)
{
	*p = T;
	*q = NULL; /* инициализация */
	s = 0;
	while(*p != NULL)
	{
		if((*p)->i == a)
			return;
		*q = *p; /* сохранение значения p */
		if(a < (*p)->i)
		{
			*p = (*p)->l; /* поиск слева */
			s = -1;
		}
		else
		{
			*p = (*p)->r; /* поиск справа */
			s = +1;
		}
	}
}
\end{lstlisting}

\end{document}
