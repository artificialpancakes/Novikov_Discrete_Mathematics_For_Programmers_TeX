\documentclass{article}

\usepackage[T2A]{fontenc}
\usepackage[utf8]{inputenc}
\usepackage[english, russian]{babel}
\usepackage[left = 2cm]{geometry}
\usepackage{listings}
\usepackage{amssymb}
\usepackage{caption}

\addto\captionsrussian{
  \renewcommand{\lstlistingname}{Алгоритм}
}
\DeclareCaptionFormat{listing}{\par\vskip1pt#1#2#3}
\captionsetup[lstlisting]{format=listing, singlelinecheck=false, labelfont=bf}

\lstset{
  language=C,
  extendedchars=\true,
  keepspaces=true,
  breaklines=true,
  frame=lines,
  aboveskip=-5pt,
  tabsize=1,
  morekeywords={Вход:, Выход:, true, false, bool, NULL, vector, adj_list, stack, vector, list}
}

\begin{document}

\vspace{5pt} \hrule
\begin{lstlisting}[caption={Построение эйлерова цикла}, label=p341_euler_cycle, escapechar=`]
`\noindent\textbf{Вход:} орграф $G(V,E)$, заданный списками смежности $\Gamma(v)$.\\`
`\textbf{Выход:} последовательность вершин эйлерова цикла.`
vector* construct_euler_cycle(int p, adj_list **G) {
  vector *out = vector_create();
  stack *S = stack_create(); /* стек для хранения вершин */
  if (p < 1) {
    return out;
  }
  stack_push(S, 0); /* положить произвольную вершину в стек */
  while (stack_size(S) != 0) {
    int v = stack_peek(S); /* v - верхний элемент стека */
    if (G[v] == NULL) {
      /* очередная вершина эйлерова цикла */
      stack_pop(S);
      vector_push_back(out, v);
    }
    else {
      int u = G[v]->v; /* взять первую вершину из списка смежности */
      stack_push(S, u); /* положить u в стек */
      adj_list_remove(G[v], u); /* удалить ребро (v, u) */
      adj_list_remove(G[u], v);
    }
  }
}
\end{lstlisting}

\end{document}
