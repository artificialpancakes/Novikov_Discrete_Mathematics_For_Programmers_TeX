\documentclass{article}

\usepackage[T2A]{fontenc}
\usepackage[utf8]{inputenc}
\usepackage[russian, english]{babel}
\usepackage[left = 2cm]{geometry}
\usepackage{listings}

\begin{document}
\lstset{language = C, extendedchars = \true, keepspaces = true}
\begin{lstlisting}
bool check_merge(elem *a, elem *b)
{
	elem *pa = a, *pb = b; //инициализация

	while(pa != NULL && pb != NULL)
	{
		if(pa->i < pb->i)
			return false; //элемент множества A отсутствует в множестве B
		else if(pa->i > pb->i)
			pb = pb->n; //элемент множества A, может быть, присутствует в
			множестве B
		else
		{
			pa = pa->n; //здесь pa->i = pb->i, то есть
			pb = pb->n; //элемент множества A точно присутствует в множестве B
		}
	}

	return pa == NULL; //true, если A исчерпано, false - в противном случае
}
\end{lstlisting}
\end{document}
