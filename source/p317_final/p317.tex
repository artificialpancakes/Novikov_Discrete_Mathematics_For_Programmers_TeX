\documentclass{article}

\usepackage[T2A]{fontenc}
\usepackage[utf8]{inputenc}
\usepackage[english, russian]{babel}
\usepackage[left = 2cm]{geometry}
\usepackage{listings}
\usepackage{amssymb}
\usepackage{caption}

\addto\captionsrussian{
  \renewcommand{\lstlistingname}{Алгоритм}
}
\DeclareCaptionFormat{listing}{\par\vskip1pt#1#2#3}
\captionsetup[lstlisting]{format=listing, singlelinecheck=false, labelfont=bf}

\lstset{
  language=C,
  extendedchars=\true,
  keepspaces=true,
  breaklines=true,
  frame=lines,
  aboveskip=-5pt,
  tabsize=1,
  morekeywords={Вход:, Выход:, true, false, bool, NULL, node}
}

\begin{document}

\vspace{5pt} \hrule
\begin{lstlisting}[caption={Вставка узла в дерево сортировки}, label=p317, escapechar=\%]
%\noindent\textbf{Вход:} дерево сортировки T, заданное указателем на корень; ключ a: key.\\%
%\textbf{Выход:} модифицированное дерево сортировки T.%
node *add_node(node *T, int a)
{
	if(T == NULL)
	{
		T = NewNode(a); /* первый узел в дереве */
		return T;
	}
	node *p = T; /* указатель на текущий узел */
	while(1)
	{
		if(a == p->i)
			return T; /* в дереве уже есть такой ключ */
		if(a < p->i)
		{
			if(p->l == NULL)
			{
				p->l = NewNode(a); /* создаем новый узел */
				return T; /* и подцепляем его к p слева */
			}
			else
				p = p->l; /* продолжаем поиск места для вставки слева */
		}
		else if(a > p->i)
		{
			if(p->r == NULL)
			{
				p->r = NewNode(a); /* создаем новый узел */
				return T; /* и подцепляем его к p справа */
			}
			else
				p = p->r; /* продолжаем поиск места для вставки справа */
		}
	}
}
\end{lstlisting}

\end{document}
