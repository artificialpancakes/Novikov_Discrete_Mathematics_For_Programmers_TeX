\vspace{5pt} \hrule
\begin{lstlisting}[caption={Проверка правильности скобочной структуры}, label=p309, escapechar=\%]
%\noindent\textbf{Вход:} строка s: \textbf{array} [1..n] \textbf{of char}, возможно, содержащая скобки '(' и ')'.\\%
%\textbf{Выход:} число 0, если скобочная структура правильна, или число в диапазоне 1..(n + 1), указывающее на позицию в строке, где скобочная структура нарушена.%
int check_bracket(int n, char s[n])
{
	int p = 0; /* число прочитанных открывающих минус число закрывающих скобок */
	for(int i = 0; i < n; ++i)
	{
		if(s[i] == '(')
			p = p + 1; /* прочли открывающую скобку */
		if(s[i] == ')')
			p = p - 1; /* прочли закрывающую скобку */
		if(p < 0)
			return i + 1; /* лишняя закрывающая скобка */
	}
	if(p == 0)
		return 0; /* скобочная структура правильна */
	else
		return n + 1; /* не хватает закрывающих скобок */
}
\end{lstlisting}
