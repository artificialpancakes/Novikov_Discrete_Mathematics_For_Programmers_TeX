\documentclass{article}

\usepackage[T2A]{fontenc}
\usepackage[utf8]{inputenc}
\usepackage[english, russian]{babel}
\usepackage[left = 2cm]{geometry}
\usepackage{listings}
\usepackage{amssymb}
\usepackage{caption}

\addto\captionsrussian{
  \renewcommand{\lstlistingname}{Алгоритм}
}
\DeclareCaptionFormat{listing}{\par\vskip1pt#1#2#3}
\captionsetup[lstlisting]{format=listing, singlelinecheck=false, labelfont=bf}

\lstset{
  language=C,
  extendedchars=\true,
  keepspaces=true,
  breaklines=true,
  frame=lines,
  aboveskip=-5pt,
  tabsize=1,
  morekeywords={Вход:, Выход:, true, false, bool, NULL}
}

\begin{document}

\vspace{5pt} \hrule
\begin{lstlisting}[caption={Определение расстояний от источника в бесконтурном графе}, label=p290, escapechar=\%]
%\noindent\textbf{Вход:} орграф G(V, E), заданный матрицей длин дуг C: \textbf{array} [1..p,1..p] \textbf{of real} и списками предшествующих узлов $\Gamma^{-1}$; источник имеет номер 1.\\%
%\textbf{Выход:} вектор T: \textbf{array} [1..p] \textbf{of real} длин кратчайших путей от источника.%
void distance_from_source(int p, int C[p][p], T[p])
{
	for(int i = 0; i < p; ++i)
		T[i] = C[1][i]; /* начальное приближение определяется матрицей */
	for(int i = 1; i < p; ++i)
	{
		for(int j = 0; j < p; ++j)
		{
			if(C[j][i] == 0 || C[j][i] == INT_MAX) continue;
			T[i] = min(T[i], T[j] + C[j][i]); /* пересчет оценки длины пути */
		}
	}
}
\end{lstlisting}

\end{document}
