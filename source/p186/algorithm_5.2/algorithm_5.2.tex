\documentclass[10pt, a5paper]{article}

\usepackage[utf8]{inputenc}
\usepackage[T2A]{fontenc}
\usepackage[russian, english]{babel}
\usepackage[left = 2cm, right = 2cm, top = 2cm, bottom = 2cm]{geometry}
\usepackage{listings}

\begin{document}
\lstset{
language = C,
extendedchars = \true,
keepspaces = true,
morekeywords = {true, false, bool, NULL},
basicstyle = \fontsize{7}{9}\selectfont,
commentstyle = \fontsize{7}{9}\selectfont,
frame = b}

\noindent\textbf{Алгоритм 5.2} Генерация перестановок в антилексикографическом порядке\\
\noindent\textbf{Вход:} n - количество элементов\\
\noindent\textbf{Выход:} последовательность перестановок элементов 1,...,n в\\
антилексикографическом порядке
\begin{lstlisting}
for(int i = 0; i < n; ++i)
  P[i] = i; /* инициализация */
Antilex(n); /* вызов рекурсивной процедуры Antilex */
\end{lstlisting}
\end{document}
