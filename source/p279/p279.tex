\documentclass{article}

\usepackage[T2A]{fontenc}
\usepackage[utf8]{inputenc}
\usepackage[english, russian]{babel}
\usepackage[left = 2cm]{geometry}
\usepackage{listings}
\usepackage{amssymb}
\usepackage{caption}

\addto\captionsrussian{
  \renewcommand{\lstlistingname}{Алгоритм}
}
\DeclareCaptionFormat{listing}{\par\vskip1pt#1#2#3}
\captionsetup[lstlisting]{format=listing, singlelinecheck=false, labelfont=bf}

\lstset{
  language=C,
  extendedchars=\true,
  keepspaces=true,
  breaklines=true,
  frame=lines,
  aboveskip=-5pt,
  tabsize=1,
  morekeywords={Вход:, Выход:, true, false, bool, NULL, flow}
}

\begin{document}

\vspace{5pt} \hrule
\begin{lstlisting}[caption={Нахождение максимального потока}, label=p279_max_flow, escapechar=\%]
%\noindent\textbf{Вход:} сеть G(V, E) с источником s и стоком t, заданная матрицей пропускных способностей C: \textbf{array} [1..p, 1..p] \textbf{of real}.\\%
%\textbf{Выход:} матрица максимального потока F: \textbf{array} [1..p, 1..p] \textbf{of real}.%
void max_flow(int p, int s, int t, int C[p][p], F[p][p])
{
	int N[p], S[p];
	flow P[p];
	for(int u = 0; u < p; ++u)
		for(int v = 0; v < p; ++v)
			F[u][v] = 0; /* вначале поток нулевой */
	M: /* итерация увелечения потока */
	for(int v = 0; v < p; ++v) /* инициализация */
	{
		S[v] = 0;
		N[v] = 0;
		P[v].s = '+';
		P[v].n = 0;
		P[v].d = 0;
	}
	S[s] = 1; /* так как $s \in S$ */
	P[s].s = '+';
	P[s].n = 0;
	P[s].d = DBL_MAX;
	do
	{
		a = 0; /* признак расширения S */
		for(int v = 0; v < p; ++v)
		{
			if(S[v] == 1 && N[v] == 0)
			{
				for(int u = 0; u < p; ++u)
				{
					if(C[v][u] == 0) continue;
					if((S[u] == 0) && (F[v][u] < C[v][u]))
	  				{
						S[u] = 1;
						P[u].s = '+';
						P[u].n = v;
						P[u].d = min(P[v].d, C[v][u] - F[v][u]);
						a = 1;
					}
				}
				for(int u = 0; u < p; ++u)
				{
					if(C[u][v] == 0) continue;
					if((S[u] == 0) && (F[u][v] > 0))
					{
						S[u] = 1;
						P[u].s = '-';
						P[u].n = v;
						P[u].d = min(P[v].d, F[u][v]);
						a = 1;
					}
				}
				N[v] = 1; /* узел v отмечен */
			}
		}
		if(S[t])
		{
			int x = t; /* текущий узел аргументальной цепи */
			int d = P[t].d; /* величина изменения потока */
			while(x != s)
			{
				if(P[x].s == '+')
					F[P[x].n][x] = F[P[x].n][x] + d; /* увеличение потока */
				else
					F[x][P[x].n] = F[x][P[x].n] - d; /* увеличение потока */
				x = P[x].n; /* предыдущий узел аргументальной цепи */
			}
			goto M;
		}
	} while(a != 0);
}
\end{lstlisting}

\end{document}
