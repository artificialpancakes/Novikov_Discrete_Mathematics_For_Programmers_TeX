\documentclass[a5paper]{article}

\usepackage[T2A]{fontenc}
\usepackage[utf8]{inputenc}
\usepackage[russian, english]{babel}
\usepackage[left = 2cm, right = 2cm, top = 2cm, bottom = 2cm]{geometry}
\usepackage{listings}

\begin{document}
\lstset{
language = C,
extendedchars = \true,
keepspaces = true,
morekeywords = {true, false, bool, elem, NULL},
basicstyle = \fontsize{7}{9}\selectfont,
commentstyle = \fontsize{7}{9}\selectfont}

\begin{lstlisting}
bool method_of_resolution(set C)
{
  set M = C; //M - текущее множество предложений
  while(1)
  {
    elem c1, c2;
    Choose(M, c1, c2, p1, p2, o); //выбор родительских предложений
    if(empty(c1) && empty(c2))
      return false; //нечего резольвировать - теорема опровергнута
    elem c = R(c1, c2, p1, p2, o); //вычисление резольвенты
    add_elem(M, c); //пополнение текущего множества
    if(empty(c)) break;
  }
  return true; //теорема доказана
}
\end{lstlisting}
\end{document}
