\documentclass{article}

\usepackage[T2A]{fontenc}
\usepackage[utf8]{inputenc}
\usepackage[english, russian]{babel}
\usepackage[left = 2cm]{geometry}
\usepackage{listings}
\usepackage{amssymb}
\usepackage{caption}

\lstset{
  language=C,
  extendedchars=\true,
  keepspaces=true,
  breaklines=true,
  aboveskip=-5pt,
  tabsize=1,
  morekeywords={Вход:, Выход:, true, false, bool, NULL}
}

\begin{document}

\begin{lstlisting}[escapechar=\%]
%\noindent\textbf{Вход:} k - номер символа в исходном тексте, начиная с которого нужно искать в тексте слова из словаря.%
%\textbf{Выход:} p - индекс самого длинного слова в словаре, совпадающего с символами f[k]..f[k + l]. Если такого слова в словаре нет, то p = 0.%
int FD(int k, char *f)
{
	int l = 0, p = 0; /* начальное состояние */
	for(int i = 0; i < d; ++i)
	{
		int m = Length(i); /* длинна слова в словаре */
		char s[m];
		for(int j = 0; j < m; ++j)
			s[j] = f[k + j];
		if((m > l) && (strcmp(D[i], s) == 0))
		{
			p = i; /* нашли более подходящее слово */
			l = m;
		}
	}
	return p;
}
\end{lstlisting}

\end{document}
