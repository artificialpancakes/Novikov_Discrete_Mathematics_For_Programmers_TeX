\vspace{5pt} \hrule
\begin{lstlisting}[caption={Распаковка по методу Лемпела - Зива}, label=p230_2, escapechar=\%]
%\noindent\textbf{Вход:} сжатый текст, представленный массивом пар g: \textbf{array} [1..m] \textbf{of record} p: \textbf{int}; q: \textbf{char end record}, где p - номер слова в словаре, q - код дополняющей буквы.\\%
%\textbf{Выход:} исходный текст, заданный последовательностью строк и символов.%
void unpack(int m, pair *g)
{
	char D[100][50];
	D[0][0] = '\0';
	int d = 0; /* начальное состояние словаря */
	for(int k = 0; k < m; ++k)
	{
		int p = g[k].p; /* p - индекс слова в словаре */
		int q = g[k].q; /* q - дополнительная буква */
		push(Res, D[p], q); /* добавление слова и еще одной буквы в вектор */
		d = d + 1;
		strcpy(D[d], D[p]);
		int l = strlen(D[d]);
		D[d][l] = q; /* пополнение словаря */
	}
}
\end{lstlisting}
