\documentclass{article}

\usepackage[T2A]{fontenc}
\usepackage[utf8]{inputenc}
\usepackage[english, russian]{babel}
\usepackage[left = 2cm]{geometry}
\usepackage{listings}
\usepackage{amssymb}
\usepackage{caption}

\addto\captionsrussian{
  \renewcommand{\lstlistingname}{Алгоритм}
}
\DeclareCaptionFormat{listing}{\par\vskip1pt#1#2#3}
\captionsetup[lstlisting]{format=listing, singlelinecheck=false, labelfont=bf}

\lstset{
  language=C,
  extendedchars=\true,
  keepspaces=true,
  breaklines=true,
  frame=lines,
  aboveskip=-5pt,
  tabsize=1,
  morekeywords={Вход:, Выход:, true, false, bool, NULL}
}

\begin{document}

\vspace{5pt} \hrule
\begin{lstlisting}[caption={Интерпретация формул - рекурсивная функция Eval}, label=p_112, escapechar=\%]
%\noindent\textbf{Вход:} формула $\mathfrak{F}$, множество F функций базиса, значения переменных $x_1, \ldots, x_n$.\\%
%\textbf{Выход:} значение формулы $\mathfrak{F}$ на значениях $x_1, \ldots, x_n$ или значение \textbf{fail}, если значение формулы не может быть определено.%
res Eval(int n, char *F_, set_func F, char *t[n], int x[n])
{
	res r;
	r.fail = false;
	for(int i = 0; i < n; ++i)
	{
		if(strcmp(F_, x_i) == 0)
		{
			r.val = x[i]; /* значение переменной задано */
			return r;
		}
	}
	if(strcmp(F_, f(t)) == 0)
	{
		if(!func_include(f, F))
		{
			r.fail = true; /* функция не входит в базис */
			return r;
		}
		res y[n];
		for(int j = 0; j < n; ++j)
		{
			y[j] = Eval(n, t[j], F, t, x); /* значение j- го аргумента */
			if(y[j].fail == true)
			{
				r.fail = true;
				return r;
			}
		}
		r.val = f(y); /* вычисленное значение главной операции */
		return r;
	}
	r.fail = true; /* это не формула */
	return r;
}
\end{lstlisting}

\end{document}
