\vspace{5pt} \hrule
\begin{lstlisting}[caption={Вычисление транзитивного замыкания отношения}, label=p_58, escapechar=\%]
%\noindent\textbf{Вход:} отношение, заданное матрицей R: \textbf{array}[1..n,1..n] \textbf{of} 0..1.\\%
%\textbf{Выход:} транзитивное замыкание отношения, заданное матрицей T: \textbf{array}[1..n,1..n] \textbf{of} 0..1.%
void warschall_algorithm(int n, int R[n][n], T[n][n])
{
	int S[n][n] = R;
	for(int i = 0; i < n; ++i)
	{
		for(int j = 0; j < n; ++j)
		{
			for(int k = 0; k < n; ++k)
				 T[j][k] = S[j][k] | (S[j][i] & S[i][k]);
		}
		S = T;
	}
}
\end{lstlisting}
