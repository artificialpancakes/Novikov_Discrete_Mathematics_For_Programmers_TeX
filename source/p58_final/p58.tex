\documentclass{article}

\usepackage[T2A]{fontenc}
\usepackage[utf8]{inputenc}
\usepackage[english, russian]{babel}
\usepackage[left = 2cm]{geometry}
\usepackage{listings}
\usepackage{amssymb}
\usepackage{caption}

\addto\captionsrussian{
  \renewcommand{\lstlistingname}{Алгоритм}
}
\DeclareCaptionFormat{listing}{\par\vskip1pt#1#2#3}
\captionsetup[lstlisting]{format=listing, singlelinecheck=false, labelfont=bf}

\lstset{
  language=C,
  extendedchars=\true,
  keepspaces=true,
  breaklines=true,
  frame=lines,
  aboveskip=-5pt,
  tabsize=1,
  morekeywords={Вход:, Выход:, true, false, bool, NULL}
}

\begin{document}

\vspace{5pt} \hrule
\begin{lstlisting}[caption={Вычисление транзитивного замыкания отношения}, label=p_58, escapechar=\%]
%\noindent\textbf{Вход:} отношение, заданное матрицей R: \textbf{array}[1..n,1..n] \textbf{of} 0..1.\\%
%\textbf{Выход:} транзитивное замыкание отношения, заданное матрицей T: \textbf{array}[1..n,1..n] \textbf{of} 0..1.%
void warschall_algorithm(int n, int R[n][n], T[n][n])
{
	int S[n][n] = R;
	for(int i = 0; i < n; ++i)
	{
		for(int j = 0; j < n; ++j)
		{
			for(int k = 0; k < n; ++k)
				 T[j][k] = S[j][k] | (S[j][i] & S[i][k]);
		}
		S = T;
	}
}
\end{lstlisting}

\end{document}
