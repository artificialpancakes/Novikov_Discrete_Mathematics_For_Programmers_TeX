\documentclass[11pt,b5paper,twoside,openany]{book}

\usepackage[utf8]{inputenc}
\usepackage[OT1]{fontenc}
\usepackage[T2A]{fontenc}
\usepackage[english, russian]{babel}
\usepackage{amsmath}
\usepackage{amsfonts}
\usepackage{amssymb}
\usepackage[left=2cm,right=2cm,top=2cm,bottom=2cm]{geometry}
\newcommand\tab[1][1cm]{\hspace*{#1}}
\usepackage{enumitem}
\usepackage{listings}
\pagestyle{headings}
\usepackage{caption}

\ifx\pdfoutput\undefined
\usepackage{graphicx}
\else
\usepackage[pdftex]{graphicx}
\fi

\addto\captionsrussian{
\renewcommand{\lstlistingname}{Алгоритм}
}

\DeclareCaptionFormat{listing}{\par\vskip1pt#1#2#3}
\captionsetup[lstlisting]{format=listing, singlelinecheck=false, labelfont=bf}

\lstset{
language=C,
extendedchars=\true,
keepspaces=true,
breaklines=true,
frame=lines,
aboveskip=-5pt,
tabsize=1,
morekeywords={Вход:, Выход:, true, false, bool, NULL, vector, list_node, result, INFINITY}
}
\begin{document}

\vspace{5pt} \hrule
\begin{lstlisting}[caption={Алгоритм Краскала}, label=p329_kruskal, escapechar=`]
`\noindent\textbf{Вход:} список $E$ рёбер графа $G$ с длинами, упорядоченный в порядке возрастания длин.\\`
`\textbf{Выход:} множество $T$ рёбер кратчайшего остова.`
vector* kruskal_algorithm(int p, edge **E) {
  vector *T = vector_create();
  int k = 0; /* номер рассматриваемого ребра */
  for (int i = 0; i < p - 1; i++) {
    while (1) {
      vector_push_back(T, E[k]);
      k++;
      if (dfs(T)) {
        break;
      }
      else {
        vector_pop_back(T);
      }
    }
  }
  return T;
}
\end{lstlisting}

\end{document}
