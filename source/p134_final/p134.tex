\documentclass{article}

\usepackage[T2A]{fontenc}
\usepackage[utf8]{inputenc}
\usepackage[russian, english]{babel}
\usepackage[left = 2cm]{geometry}
\usepackage{listings}
\usepackage{amssymb}
\usepackage{caption}

\renewcommand{\lstlistingname}{Алгоритм}
\DeclareCaptionFormat{listing}{\par#1#2#3}
\captionsetup[lstlisting]{format=listing, singlelinecheck=false, labelfont=bf}

\lstset{
  language=C,
  extendedchars=\true,
  keepspaces=true,
  breaklines=true,
  frame=lines,
  aboveskip=-5pt,
  tabsize=1,
  morekeywords={Вход:, Выход:, true, false, NULL, bool, vector, bitset}
}

\begin{document}

\vspace{5pt} \hrule
\begin{lstlisting}[caption={Вычисление номера кортежа в установленном порядке}, label=p134_calc_cortege_number, escapechar=`]
`\noindent\textbf{Вход:} кортеж $x$ : array [1..$n$] of 0..1 значений переменных.\\`
`\textbf{Выход:} номер $d$ кортежа $x$ при перечислении кортежей в установленном порядке.`
int calc_cortege_number(int n, int *x) {
  int d = 0; /* начальное значение индекса */
  for (int i = 0; i < n; i++) {
    d = d << 2;
    if (x[i] == 1) {
      d++; /* добавляем 1, если нужнo */
    }
  }
}
\end{lstlisting}

\end{document}
