\vspace{5pt} \hrule
\begin{lstlisting}[caption={Улучшенный алгоритм последовательного раскрашивания}, label=p360, escapechar=\%]
%\noindent\textbf{Вход:} граф G.\\%
%\textbf{Выход:} раскраска графа - массив C: \textbf{array} [1..p] \textbf{of} 1..p.%
void improved_coloring_graph(int p, int G[p][p], int V[p], int C[p])
{
	sort(p, V); /* упорядочить вершины по невозрастанию степени */
	int c = 1; /* первый цвет */
	for(int v = 0; v < p; ++v)
		C[v] = 0; /* все не раскрашены */
	while(!empty(p, V))
	{
		for(int v = 0; v < p; ++p)
		{
			int flag = 0;
			for(int u = 0; u < p; ++u)
			{
				if(G[v][u] == 0) continue;
				if(C[u] == c) flag = 1; /* вершину v нельзя покрасить в цвет c */
			}
			if(flag) continue;
			C[v] = c; /* красим вершину v в цвет c */
			V[v] = -1; /* и удаляем ее из рассмотрения */
		}
		c = c + 1; /* следующий цвет */
	}
}
\end{lstlisting}
