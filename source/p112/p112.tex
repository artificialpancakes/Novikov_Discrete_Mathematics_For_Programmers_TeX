\documentclass[a5paper]{article}

\usepackage[T2A]{fontenc}
\usepackage[utf8]{inputenc}
\usepackage[russian, english]{babel}
\usepackage[left = 2cm, right = 2cm, top = 2cm, bottom = 2cm]{geometry}
\usepackage{listings}

\begin{document}
\lstset{
language = C,
extendedchars = \true,
keepspaces = true,
morekeywords = {true, false, bool, elem, NULL},
basicstyle = \fontsize{7}{9}\selectfont,
commentstyle = \fontsize{7}{9}\selectfont}
\begin{lstlisting}
res Eval(int n, char *F_, set_func F, char *t[n], int x[n])
{
  res r;
  r.fail = false;
  for(int i = 0; i < n; ++i)
  {
    if(strcmp(F_, x_i) == 0)
    {
      r.val = x[i]; //значение переменной задано
      return r;
    }
  }
  if(strcmp(F_, f(t)) == 0)
  {
    if(!func_include(f, F))
    {
      r.fail = true; //функция не входит в базис
      return r;
    }
    res y[n];
    for(int j = 0; j < n; ++j)
    {
      y[j] = Eval(n, t[j], F, t, x); //значение j- го аргумента
      if(y[j].fail == true)
      {
        r.fail = true;
        return r;
      }
    }
    r.val = f(y); //вычисленное значение главной операции
    return r;
  }
  r.fail = true; //это не формула
  return r;
}
\end{lstlisting}
\end{document}
