\vspace{5pt} \hrule
\begin{lstlisting}[caption={Алгоритм Прима}, label=p330, escapechar=\%]
%\noindent\textbf{Вход:} граф G(V, E), заданный матрицей длин ребер C.\\%
%\textbf{Выход:} множество T ребер кратчайшего остова.%
void algorithm_prima(int p, int C[p][p], set T)
{
	int a[p], b[p];
	int u = rand() %\%% p; /* выбираем произвольную вершину */
	set S = create();
	S = add(S, u); /* S - множество вершин, включенных в кратчайший остов */
	T = empty(); /* T - множество ребер, включенных в кратчайший остов */
	for(int v = 0; v < p; ++v)
	{
		if(v == u) continue;
		if(C[u][v] != 0)
		{
			a[v] = u; /* u - ближайшая вершина остова */
			b[v] = C[u][v]; /* C[u][v] - длина соответствующего ребра */
		}
		else
		{
			a[v] = 0; /* ближайшая вершина остова неизвестна */
			b[v] = INT_MAX; /* и расстояние также неизвестно */
		}
	}
	for(int i = 0; i < p - 1; ++i)
	{
		int x = INT_MAX, w; /* начальное значение для поиска ближайшей вершины */
		for(int v = 0; v < p; ++v)
		{
			if(include(S, v)) continue;
			if(b[v] < x)
			{
				w = v; /* нашли более близкую вершину */
				x = b[v]; /* и расстояние до нее */
			}
		}
		S = add(S, w); /* добавляем найденную вершину в остов */
		T = add(T, a[w], w); /* добавляем найденное ребро в остов */
		for(int v = 0; v < p; ++v)
		{
			if(C[v][w] == 0) continue;
			if(!include(S, v))
			{
				if(b[v] > C[v][w])
				{
					a[v] = w; /* изменяем ближайшую вершину остова */
					b[v] = C[v][w]; /* и длину ведущего к ней ребра */
				}
			}
		}
	}
}
\end{lstlisting}
