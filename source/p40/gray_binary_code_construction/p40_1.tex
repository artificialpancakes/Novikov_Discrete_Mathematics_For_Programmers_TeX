
\documentclass{article}

\usepackage[T2A]{fontenc}
\usepackage[utf8]{inputenc}
\usepackage[russian, english]{babel}
\usepackage[left = 2cm]{geometry}
\usepackage{listings}

\renewcommand{\lstlistingname}{Алгоритм}

\begin{document}

\lstset{language = C,
    extendedchars = \true,
    keepspaces = true,
    breaklines=true,
    frame=lines}
\hrule
\begin{lstlisting}[title=\textbf {Алгоритм 1.2} Построение бинарного кода Грея]
vector* gray_binary_code(int n) { /* n >= 0 - мощность множества~*/
  vector *out = vector_create(); /* последовательность кодов подмножеств B~*/
  bitset *B = bitset_create(n); /* битовая шкала для представления подмножеств~*/
  for (int i = 0; i < n; i++) {
    bitset_set(B, i, 0); /* инициализация~*/
  }
  vector_push_back(out, B); /* пустое множество~*/

  for (int i = 1; i < pow(2, n); i++) {
    int p = Q(i); /* определение номера элемента, подлежащего добавлению или удалению~*/
    bitset_set(B, p, 1 - bitset_get(B, p)); /* добавление или удаление элемента~*/
    vector_push_back(out, B); /* добавление или удаление элемента~*/
  }

  return out;
}
\end{lstlisting}
\end{document}
