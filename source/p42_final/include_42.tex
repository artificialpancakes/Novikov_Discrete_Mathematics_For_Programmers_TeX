\vspace{5pt} \hrule
\begin{lstlisting}[caption={Проверка включения слиянием}, label=p_42, escapechar=\%]
%\noindent\textbf{Вход:} Проверяемые множества A и B, которые заданы указателями a и b.\\%
%\textbf{Выход:} \textbf{true}, если A $\subset$ B, в противном случае \textbf{false}.%
bool check_inclusion(elem *a, elem *b)
{
	elem *pa = a, *pb = b; /* инициализация */
	while(pa != NULL && pb != NULL)
	{
		if(pa->i < pb->i)
			return false; /* элемент множества A отсутствует в множестве B */
		else if(pa->i > pb->i)
			pb = pb->n; /* элемент множества A, может быть, присутствует в множестве B */
		else
		{
			pa = pa->n; /* здесь pa->i = pb->i, то есть */
			pb = pb->n; /* элемент множества A точно присутствует в множестве B */
		}
	}
	return pa == NULL; /* true, если A исчерпано, false - в противном случае */
}
\end{lstlisting}
