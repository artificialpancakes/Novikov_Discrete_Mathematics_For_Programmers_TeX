\documentclass{article}

\usepackage[T2A]{fontenc}
\usepackage[utf8]{inputenc}
\usepackage[english, russian]{babel}
\usepackage[left = 2cm]{geometry}
\usepackage{listings}
\usepackage{amssymb}
\usepackage{caption}

\addto\captionsrussian{
  \renewcommand{\lstlistingname}{Алгоритм}
}
\DeclareCaptionFormat{listing}{\par\vskip1pt#1#2#3}
\captionsetup[lstlisting]{format=listing, singlelinecheck=false, labelfont=bf}

\lstset{
  language=C,
  extendedchars=\true,
  keepspaces=true,
  breaklines=true,
  frame=lines,
  aboveskip=-5pt,
  tabsize=1,
  morekeywords={Вход:, Выход:, true, false, bool, NULL}
}

\begin{document}

\vspace{5pt} \hrule
\begin{lstlisting}[caption={Алгоритм последовательного раскрашивания}, label=p359, escapechar=\%]
%\noindent\textbf{Вход:} граф G.\\%
%\textbf{Выход:} раскраска графа - массив C: \textbf{array} [1..p] \textbf{of} 1..p.%
void coloring_graph(int p, int G[p][p], int C[p])
{
	for(int v = 0; v < p; ++v)
		C[v] = 0; /* все вершины не раскрашены */
	for(int v = 0; v < p; ++v)
	{
		int A[p];
		for(int i = 0; i < p; ++i)
			A[p] = p; /* все цвета */
		for(int u = 0; u < p; ++u)
		{
			if(G[v][u] == 0) continue;
			for(int i = 0; i < p; ++i)
			{
				if(A[i] == C[u])
					A[i] = -1; /* занятые для вершины v цвета */
			}
		}
		C[v] = min(p, A); /* минимальный свободный цвет */
	}
}
\end{lstlisting}

\end{document}
