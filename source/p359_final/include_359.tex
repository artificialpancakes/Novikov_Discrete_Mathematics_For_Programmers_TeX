\vspace{5pt} \hrule
\begin{lstlisting}[caption={Алгоритм последовательного раскрашивания}, label=p359, escapechar=\%]
%\noindent\textbf{Вход:} граф G.\\%
%\textbf{Выход:} раскраска графа - массив C: \textbf{array} [1..p] \textbf{of} 1..p.%
void coloring_graph(int p, int G[p][p], int C[p])
{
	for(int v = 0; v < p; ++v)
		C[v] = 0; /* все вершины не раскрашены */
	for(int v = 0; v < p; ++v)
	{
		int A[p];
		for(int i = 0; i < p; ++i)
			A[p] = p; /* все цвета */
		for(int u = 0; u < p; ++u)
		{
			if(G[v][u] == 0) continue;
			for(int i = 0; i < p; ++i)
			{
				if(A[i] == C[u])
					A[i] = -1; /* занятые для вершины v цвета */
			}
		}
		C[v] = min(p, A); /* минимальный свободный цвет */
	}
}
\end{lstlisting}
