\documentclass{article}

\usepackage[T2A]{fontenc}
\usepackage[utf8]{inputenc}
\usepackage[english, russian]{babel}
\usepackage[left = 2cm]{geometry}
\usepackage{listings}
\usepackage{amssymb}
\usepackage{caption}

\addto\captionsrussian{
  \renewcommand{\lstlistingname}{Алгоритм}
}
\DeclareCaptionFormat{listing}{\par\vskip1pt#1#2#3}
\captionsetup[lstlisting]{format=listing, singlelinecheck=false, labelfont=bf}

\lstset{
  language=C,
  extendedchars=\true,
  keepspaces=true,
  breaklines=true,
  frame=lines,
  aboveskip=-5pt,
  tabsize=1,
  morekeywords={Вход:, Выход:, true, false, bool, NULL, vector, formula}
}

\begin{document}

\vspace{5pt} \hrule
\begin{lstlisting}[caption={Рекурсивная функция Unify}, label=p152_unify, escapechar=`]
`\noindent\textbf{Вход:} формулы $A$ и $B$.\\`
`\textbf{Выход:} \textbf{false}, если формулы не унифицируемы, или \textbf{true} и наиболее общий унификатор $S$.`
enum formula_type { IMPL, NEG, VAR };
bool Unify(formula *A, formula *B) {
  if (f(A) == VAR) {
    char *v = A->var; /* переменная */
    formula *permutation = list_get(S, v);
    if (permutation == NULL) {
      list_set(S, v, B); /* то есть добавляем подстановку {B//v} */
    }
    else {
      /* либо эта подстановка уже есть, либо унификация невозможна */
      return permutation == B;
    }
  }
  if (f(A) != f(B)) {
    /* главные операции различны — унификация невозможна */
    return false;
  }
  if (f(A) == NEG) {
    /* пытаемся унифицировать операнды отрицаний */
    return Unify(r(A), r(B));
  }
  if (f(A) == IMPL) {
    /* пытаемся унифицировать операнды импликаций */
    return Unify(l(A), l(B)) && Unify(r(A), r(B));
  }
}
\end{lstlisting}

\end{document}
