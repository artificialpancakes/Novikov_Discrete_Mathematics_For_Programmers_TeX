\documentclass{article}

\usepackage[T2A]{fontenc}
\usepackage[utf8]{inputenc}
\usepackage[english, russian]{babel}
\usepackage[left = 2cm]{geometry}
\usepackage{listings}
\usepackage{amssymb}
\usepackage{caption}

\addto\captionsrussian{
  \renewcommand{\lstlistingname}{Алгоритм}
}
\DeclareCaptionFormat{listing}{\par\vskip1pt#1#2#3}
\captionsetup[lstlisting]{format=listing, singlelinecheck=false, labelfont=bf}

\lstset{
  language=C,
  extendedchars=\true,
  keepspaces=true,
  breaklines=true,
  frame=lines,
  aboveskip=-5pt,
  tabsize=1,
  morekeywords={Вход:, Выход:, true, false, bool, NULL}
}

\begin{document}

\vspace{5pt} \hrule
\begin{lstlisting}[caption={Генерация перестановок в антилексикографическом порядке}, label=p186_2, escapechar=\%]
%\noindent\textbf{Вход:} n - количество элементов.\\%
%\textbf{Выход:} последовательность перестановок элементов $1, \ldots, n$ в антилексикографическом порядке.%
for(int i = 0; i < n; ++i)
	P[i] = i; /* инициализация */
Antilex(n); /* вызов рекурсивной процедуры Antilex */
\end{lstlisting}

\end{document}
