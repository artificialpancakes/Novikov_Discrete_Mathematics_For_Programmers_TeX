\vspace{5pt} \hrule
\begin{lstlisting}[caption={Сортировка методом пузырька}, label=p186, escapechar=`]
`\noindent\textbf{Вход:} массив типа $B A[n]$, где значения элементов массива расположены в произвольном порядке и для значений типа $B$ задано отношение $<$.\\`
`\textbf{Выход:} массив типа$B A[n]$, в котором значения расположены в порядке возрастания.`
for(int i = 1; i < n; i++)
    for(int j = n - 1; j >= i; j--)
        if(A[j] < A[j - 1]){
            A[j - 1] += A[j];
            A[j] = A[j - 1] - A[j]; /* транспозиция соседних элементов */
            A[j - 1] -= A[j];
        }
\end{lstlisting}
