\documentclass{article}

\usepackage[T2A]{fontenc}
\usepackage[utf8]{inputenc}
\usepackage[english, russian]{babel}
\usepackage[left = 2cm]{geometry}
\usepackage{listings}
\usepackage{amssymb}
\usepackage{caption}

\addto\captionsrussian{
  \renewcommand{\lstlistingname}{Алгоритм}
}
\DeclareCaptionFormat{listing}{\par\vskip1pt#1#2#3}
\captionsetup[lstlisting]{format=listing, singlelinecheck=false, labelfont=bf}

\lstset{
  language=C,
  extendedchars=\true,
  keepspaces=true,
  breaklines=true,
  frame=lines,
  aboveskip=-5pt,
  tabsize=1,
  morekeywords={Вход:, Выход:, true, false, bool, NULL, node}
}

\begin{document}

\vspace{5pt} \hrule
\begin{lstlisting}[caption={Удаление узла из дерева сортировки}, label=p319, escapechar=\%]
%\noindent\textbf{Вход:} дерево сортировки T, заданное указателем на корень; ключ a: key.\\%
%\textbf{Выход:} модифицированное дерево сортировки T.%
node *remove_node(node *T, int a)
{
	node *p, *q;
	int s;
	Find(T, a, &p, &q, s); /* поиск удаляемого узла */
	if(p == NULL)
		return T; /* нет такого узла - ничего делать не нужно */
	if(p->r == NULL)
		Delete(p, q, p->l, s); /* случай 1, рис. 9.12, слева */
	else
	{
		node *u = p->r;
		if(u->l == NULL)
		{
			u->l = p->l;
			Delete(p, q, u, s); /* случай 2, рис. 9.12, в центре */
		}
		else
		{
			node *w = u, *v = u->l;
			while(v->l != NULL)
			{
				w = v;
				v = v->l;
			}
			p->i = v->i;
			Delete(v, w, v->r, -1); /* случай 3, рис. 9.12, справа */
		}
	}
	return T;
}
\end{lstlisting}

\end{document}
