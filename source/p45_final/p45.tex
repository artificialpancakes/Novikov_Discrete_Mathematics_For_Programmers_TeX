\documentclass{article}

\usepackage[T2A]{fontenc}
\usepackage[utf8]{inputenc}
\usepackage[english, russian]{babel}
\usepackage[left = 2cm]{geometry}
\usepackage{listings}
\usepackage{amssymb}
\usepackage{caption}

\addto\captionsrussian{
  \renewcommand{\lstlistingname}{Алгоритм}
}
\DeclareCaptionFormat{listing}{\par\vskip1pt#1#2#3}
\captionsetup[lstlisting]{format=listing, singlelinecheck=false, labelfont=bf}

\lstset{
  language=C,
  extendedchars=\true,
  keepspaces=true,
  breaklines=true,
  frame=lines,
  aboveskip=-5pt,
  tabsize=1,
  morekeywords={Вход:, Выход:, true, false, bool, NULL, elem}
}

\begin{document}

\vspace{5pt} \hrule
\begin{lstlisting}[caption={Вычисление пересечения слиянием}, label=p_45, escapechar=\%]
%\noindent\textbf{Вход:} пересекаемые множества A и B, заданные указателями a и b.\\%
%\textbf{Выход:} пересечение $C = A \cap B$, заданное указателем c.%
elem *calc_intersection(elem *a, elem *b)
{
	elem *pa = a, *pb = b, *c = NULL, *e = NULL; /* инициализация */
	while(pa != NULL && pb != NULL)
	{
		if(pa->i < pb->i)
			pa = pa->n; /* элемент множества A не принадлежит пересечению */
		else if(pa->i > pb->i)
			pb = pb->n; /* элемент множества B не принадлежит пересечению */
		else
		{
			/* здесь pa->i = pb->i - данный элемент принадлежит пересечению */
			Append(c, e, pa->i) /* добавление элемента */
			pa = pa->n;
			pb = pb->n;
		}
	}
	return c;
}
\end{lstlisting}

\end{document}
