\documentclass[10pt, a5paper]{article}

\usepackage[utf8]{inputenc}
\usepackage[T2A]{fontenc}
\usepackage[russian, english]{babel}
\usepackage[left = 2cm, right = 2cm, top = 2cm, bottom = 2cm]{geometry}
\usepackage{listings}

\begin{document}
\lstset{
language = C,
extendedchars = \true,
keepspaces = true,
morekeywords = {true, false, bool, NULL},
basicstyle = \fontsize{7}{9}\selectfont,
commentstyle = \fontsize{7}{9}\selectfont}

\noindent\textbf{Вход:} m - параметр процедуры - количество первых элементов массива P,
для которых генерируются перестановки\\
\noindent\textbf{Выход:} последовательность перестановок 1,...,m в антилексикографическом порядке
\begin{lstlisting}
void Antilex(int m)
{
  if(m == 1)
    push(Res, P); /* добавление очередной перестановки в вектор */
  else
  {
    for(int i = 0; i < m; ++i)
    {
      Antilex(m - 1); /* рекурсивный вызов */
      if(i < m - 1)
      {
        swap(i, m - 1); /* следующий элемент */
        Reverse(m - 2); /* изменение порядка элементов */
      }
    }
  }
}
\end{lstlisting}
\end{document}
