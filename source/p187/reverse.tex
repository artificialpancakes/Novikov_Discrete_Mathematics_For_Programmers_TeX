\documentclass[10pt, a5paper]{article}

\usepackage[utf8]{inputenc}
\usepackage[T2A]{fontenc}
\usepackage[russian, english]{babel}
\usepackage[left = 2cm, right = 2cm, top = 2cm, bottom = 2cm]{geometry}
\usepackage{listings}

\begin{document}
\lstset{
language = C,
extendedchars = \true,
keepspaces = true,
morekeywords = {true, false, bool, NULL},
basicstyle = \fontsize{7}{9}\selectfont,
commentstyle = \fontsize{7}{9}\selectfont}

\noindent\textbf{Вход:} k - номер элемента, задающий отрезок массива P,
подлежащий перестановке в обратном порядке\\
\noindent\textbf{Выход:} первые k элементов массива P переставлены в
обратном порядке
\begin{lstlisting}
void Reverse(int k)
{
  int j = 0; /* нижняя граница обращаемого диапазона */
  while(j < k)
  {
    swap(j, k); /* меняем местами элементы */
    j = j + 1; /* увеличиваем нижнюю границу */
    k = k - 1; /* уменьшаем верхнюю границу */
  }
}
\end{lstlisting}
\end{document}
