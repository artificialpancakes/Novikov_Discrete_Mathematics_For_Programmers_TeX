\documentclass{article}

\usepackage[T2A]{fontenc}
\usepackage[utf8]{inputenc}
\usepackage[russian, english]{babel}
\usepackage[left = 2cm]{geometry}
\usepackage{listings}

\renewcommand{\lstlistingname}{Алгоритм}

\begin{document}

\lstset{language = C,
    extendedchars = \true,
    keepspaces = true,
    breaklines=true,
    frame=lines}
\hrule
\begin{lstlisting}[title=\textbf {Алгоритм 1.5} Вычисление объединения слиянием]
elem* calc_join(elem *a, elem *b) {
  elem *pa = a, *pb = b, *c = NULL, *e = NULL; /* инициализация~*/

  while ((pa != NULL) && (pb != NULL)) {
    int d;
    if (pa->i < pb->i) {
      d = pa->i; /* добавлению подлежит элемент множества А~*/
      pa = pa->n;
    }
    else if (pa->i > pb->i) {
      d = pb->i; /* добавлению подлежит элемент множества В~*/
      pb = pb->n;
    }
    else {
      d = pa->i; /* здесь pa.i = pb.i, и можно взять любой из элементов~*/
      pa = pa->n; pb = pb->n; /* продвижение в обоих множествах~*/
    }
    append(&c, &e, d); /* добавление элемента d в конец списка с~*/
  }

  elem *p = NULL; /* указатель «хвоста»~*/
  if (pa != NULL) {
    p = pa; /* нужно добавить в результат оставшиеся элементы множества А~*/
  }
  if (pb != NULL) {
    p = pb; /* нужно добавить в результат оставшиеся элементы множества В~*/
  }
  while (p != NULL) {
    append(&c, &e, p->i); /* добавление элемента pa.i в конец списка с~*/
    p = p->n; /* продвижение указателя «хвоста»~*/
  }
  return c;
}
\end{lstlisting}
\end{document}
