\vspace{5pt} \hrule
\begin{lstlisting}[caption={Алгоритм симметричного обхода бинарного дерева}, label=p_313, escapechar=\%]
%\noindent\textbf{Вход:} бинарное дерево, представленное списочной структурой, r - указатель на корень.\\%
%\textbf{Выход:}. последовательность узлов бинарного дерева в порядке симметричного обхода%
steck* T = create(); /* вначале стек пуст */
node* p = r; /* p указывает на корень дерева */
while(1){ /* анализируем узел, на который указывает p */
		if(p == NULL){
			if(is_empty(T))
				break; /* обход завершен */
			else{
				p = pop(T); /* левое поддерево обойдено */
				yield(p); /* очередной узел при симметричном обходе */
				p = p->right; /* начинаем обход правого поддерева */
			}
		}
		else{
			push(T, p); /* запоминаем текущий узел...*/
			p = p->left; /*...и начинаем обход левого поддерева*/
		}
	}
\end{lstlisting}
