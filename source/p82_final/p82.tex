\documentclass{article}

\usepackage[T2A]{fontenc}
\usepackage[utf8]{inputenc}
\usepackage[english, russian]{babel}
\usepackage[left = 2cm]{geometry}
\usepackage{listings}
\usepackage{amssymb}
\usepackage{caption}

\lstset{
  language=C,
  extendedchars=\true,
  keepspaces=true,
  breaklines=true,
  aboveskip=-5pt,
  tabsize=1,
  morekeywords={Вход:, Выход:, true, false, bool, NULL}
}

\begin{document}

\begin{lstlisting}[escapechar=\%]
%\noindent\textbf{Вход:} входные слова $\alpha$: \textbf{array} [1..s] \textbf{of} \{a, b\} и $\beta$: \textbf{array} [1..t] \textbf{of} \{a, b\}.%
%\textbf{Выход:} значение выражения $\alpha = \beta$ в полугруппе $\rho$.%
bool check_equality_word(int s, int t, char a[s], char b[t])
{
	if(a[0] != b[0])
		return false; /* первые буквы не совпадают */
	int N_a = 0; /* счетчик изменений буквы в a */
	for(int i = 1; i < s; ++i)
	{
		if(a[i] != a[i - 1])
			N_a = N_a + 1; /* буква изменилась */
	}
	int N_b = 0; /* счетчик изменений буквы в b */
	for(int i = 1; i < t; ++i)
	{
		if(b[i] != b[i - 1])
			N_b = N_b + 1; /* буква изменилась */
	}
	return N_a == N_b; /* слова равны, если значения счетчиков одинаковы */
}
\end{lstlisting}

\end{document}
