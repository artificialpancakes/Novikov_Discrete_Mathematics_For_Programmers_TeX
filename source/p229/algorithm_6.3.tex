\documentclass[10pt, a5paper]{article}

\usepackage[utf8]{inputenc}
\usepackage[T2A]{fontenc}
\usepackage[russian, english]{babel}
\usepackage[left = 2cm, right = 2cm, top = 2cm, bottom = 2cm]{geometry}
\usepackage{listings}

\begin{document}
\lstset{
language = C,
extendedchars = \true,
keepspaces = true,
morekeywords = {true, false, bool, NULL},
basicstyle = \fontsize{7}{9}\selectfont,
commentstyle = \fontsize{7}{9}\selectfont,
frame = b}

\noindent\textbf{Алгоритм 6.3} Упаковка по методу Лемпела - Зива\\
\noindent\textbf{Вход:} исходный текст, заданный массивом кодов символов f:
\textbf{array} [1..n] of \textbf{char}\\
\noindent\textbf{Выход:} сжатый текст, представленный последовательностью пар <p, q>,
где p - номер слова в словаре, q - код дополняющей буквы
\begin{lstlisting}
void pack(int n, char *f)
{
  D[0][0] = '\0';
  d = 0; /* начальное состояние словаря */
  int k = 0; /* номер текущей буквы в исходном тексте */
  while(k < n)
  {
    int p = FD(k, f); /* индекс найденного слова в словаре */
    int l = Length(p); /* l - длинна найденного слова в словаре */
    push(Res, p, f[k + l]);
    /* код найденного слова и еще одна буквы добавляются в вектор */
    d = d + 1;
    strcpy(D[d], D[p]);
    D[d][l] = f[k + l]; /* пополнение словаря */
    k = k + l + 1; /* продвижение вперед по исходному тексту */
  }
}
\end{lstlisting}
\end{document}
