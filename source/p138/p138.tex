\documentclass{article}

\usepackage[T2A]{fontenc}
\usepackage[utf8]{inputenc}
\usepackage[russian, english]{babel}
\usepackage[left = 2cm]{geometry}
\usepackage{listings}

\renewcommand{\lstlistingname}{Алгоритм}

\begin{document}

\lstset{language = C,
    extendedchars = \true,
    keepspaces = true,
    breaklines=true,
    frame=lines}
\hrule
\begin{lstlisting}[title=\textbf {Алгоритм 3.5} Вычисление значения функции по сокращённому дереву решений]
bool calc_func(tree_node *T, int n, int *x) {
  for (int i = 0; i < n; i++) {
    if (T->l == NULL && T->r == NULL) {
      return T->i; /* листовой узел — возвращаем значениe~*/
    }
    else {
      if (x[i]) {
        T = T->r; /* 1 — переход вправо~*/
      } else {
        T = T->l; /* 0 — переход влево~*/
      }
    }
  }
}
\end{lstlisting}
\end{document}
