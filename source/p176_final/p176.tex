\documentclass{article}

\usepackage[T2A]{fontenc}
\usepackage[utf8]{inputenc}
\usepackage[english, russian]{babel}
\usepackage[left = 2cm]{geometry}
\usepackage{listings}
\usepackage{amssymb}
\usepackage{caption}

\addto\captionsrussian{
  \renewcommand{\lstlistingname}{Алгоритм}
}
\DeclareCaptionFormat{listing}{\par\vskip1pt#1#2#3}
\captionsetup[lstlisting]{format=listing, singlelinecheck=false, labelfont=bf}

\lstset{
  language=C,
  extendedchars=\true,
  keepspaces=true,
  breaklines=true,
  frame=lines,
  aboveskip=-5pt,
  tabsize=1,
  morekeywords={Вход:, Выход:, true, false, bool, NULL, set, elem}
}

\begin{document}

\vspace{5pt} \hrule
\begin{lstlisting}[caption={Метод резолюций}, label=p_176, escapechar=\%]
%\noindent\textbf{Вход:} множество предложений C, полученных из множества формул S и формулы $\neg G$.\\%
%\textbf{Выход:} \textbf{true} - если G выводимо из S, \textbf{false} - в противном случае.%
bool method_of_resolution(set C)
{
	set M = C; /* M - текущее множество предложений */
	while(1)
	{
		elem c1, c2;
		Choose(M, c1, c2, p1, p2, o); /* выбор родительских предложений */
		if(empty(c1) && empty(c2))
			return false; /* нечего резольвировать - теорема опровергнута */
		elem c = R(c1, c2, p1, p2, o); /* вычисление резольвенты */
		add_elem(M, c); /* пополнение текущего множества */
		if(empty(c)) break;
	}
	return true; /* теорема доказана */
}
\end{lstlisting}

\end{document}
