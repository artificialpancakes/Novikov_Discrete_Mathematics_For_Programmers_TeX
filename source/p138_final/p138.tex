\documentclass{article}

\usepackage[T2A]{fontenc}
\usepackage[utf8]{inputenc}
\usepackage[russian]{babel}
\usepackage[left = 2cm]{geometry}
\usepackage{listings}
\usepackage{amssymb}
\usepackage{caption}

\addto\captionsrussian{
  \renewcommand{\lstlistingname}{Алгоритм}
}
\DeclareCaptionFormat{listing}{\par#1#2#3}
\captionsetup[lstlisting]{format=listing, singlelinecheck=false, labelfont=bf}

\lstset{
  language=C,
  extendedchars=\true,
  keepspaces=true,
  breaklines=true,
  frame=lines,
  aboveskip=-5pt,
  tabsize=1,
  morekeywords={Вход:, Выход:, true, false, NULL, bool, vector, bitset, tree_node}
}

\begin{document}

\vspace{5pt} \hrule

\begin{lstlisting}[caption={Вычисление сокращённого дерева решений}, label=p138_calc_function, escapechar=`]
`\noindent\textbf{Вход:} указатель $T$: $\uparrow$ $N$ на корень дерева решений;\\ массив $x$ : \textbf{array} [1 ..$n$] of 0..1 значений переменных.\\`
`\textbf{Выход:} 0..1 — значение булевой функции.`
bool calc_func(tree_node *T, int n, int *x) {
  for (int i = 0; i < n + 1; i++) {
    if (T->l == NULL && T->r == NULL) {
      return T->i; /* листовой узел — возвращаем значениe */
    }
    else {
      if (x[i]) {
        T = T->r; /* 1 — переход вправо */
      } else {
        T = T->l; /* 0 — переход влево */
      }
    }
  }
}
\end{lstlisting}

\end{document}
