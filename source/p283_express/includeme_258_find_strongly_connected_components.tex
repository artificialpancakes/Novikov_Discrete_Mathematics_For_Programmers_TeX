\vspace{5pt} \hrule
\begin{lstlisting}[caption={Выделение компонент сильной связности}, label=p283_scc, escapechar=`]
`\noindent\textbf{Вход:} орграф $G(V,E)$, заданный списками смежности $\Gamma(v)$.\\`
`\textbf{Выход:} последовательность вершин обхода.`
list* find_strongly_connected_components(int p, adj_list **G) {
  stack *T = stack_create();
  list *C = list_create();
  adj_list *M[p];
  int e[p];
  /* используем этот указатель, чтобы пометить вершину как удалённую из `$V$` */
  adj_list *deleted = malloc(sizeof(adj_list));
  for (int v = 0; v < p; v++) {
     /* `$M[v]$` список узлов, входящих в ту же КСС, что и `$v$` */
    adj_list_add(M[v], v);
    e[v] = 0; /* все узлы не рассмотрены */
  }
  for (int i = 0; i < p; i++) {
    if (G[i] == deleted) {
      continue; /* узел удалён */
    }
    int v = i;
    stack_push(T, v);
    e[v] = 1;
    /* SCC; вызов процедуры КСС */
  }
  free(deleted);
  stack_destroy(T);
  return C;
}
\end{lstlisting}
