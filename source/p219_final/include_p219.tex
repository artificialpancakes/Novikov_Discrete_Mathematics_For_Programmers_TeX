\vspace{5pt} \hrule
\begin{lstlisting}[caption={Построение оптимальной схемы - рекурсивная процедура Huffman}, label=p_219, escapechar=\%]
%\noindent\textbf{Вход:} $n$ количество букв, $double P[n]$ - массив вероятностей букв, упорядоченный по убыванию.\\%
%\textbf{Выход:} $bool c[n][L]$ - массив элементарных кодов, $int l[n]$ - массив длин элементарных кодов схемы оптимального префиксного кодирования.%
int Huffman(double *P, bool** C_arr, int *l, int n){
	int *C = (int*)C_arr;
	double q;
	int j;
	if(n == 1){
        C[0 * size + 0] = 0, l[0] = 1; /* первый элемент */
        C[1 * size + 0] = 1, l[1] = 1; /* второй элемент */
  }
  else{
        q = P[n - 1] + P[n]; /* сумма двух последних вероятностей */
        j = Up(P, n, q); /* поиск места и вставка */
        Huffman(P, (int**)C, l, n - 1); /* рекурсивный вызов */
        Down((int**)C, l, n, j); /* достраивание кодов */
        }
}
\end{lstlisting}
