\documentclass{article}

\usepackage[T2A]{fontenc}
\usepackage[utf8]{inputenc}
\usepackage[english, russian]{babel}
\usepackage[left = 2cm]{geometry}
\usepackage{listings}
\usepackage{amssymb}
\usepackage{caption}

\lstset{
  language=C,
  extendedchars=\true,
  keepspaces=true,
  breaklines=true,
  aboveskip=-5pt,
  tabsize=1,
  morekeywords={Вход:, Выход:, true, false, bool, NULL}
}

\begin{document}

\begin{lstlisting}[escapechar=\%]
%\noindent\textbf{Вход:} m - параметр процедуры - количество первых элементов массива P, для которых генерируются перестановки.%
%\textbf{Выход:} последовательность перестановок $1, \ldots ,m$ в антилексикографическом порядке.%
void Antilex(int m)
{
	if(m == 1)
		push(Res, P); /* добавление очередной перестановки в вектор */
	else
	{
		for(int i = 0; i < m; ++i)
		{
			Antilex(m - 1); /* рекурсивный вызов */
			if(i < m - 1)
			{
				swap(i, m - 1); /* следующий элемент */
				Reverse(m - 2); /* изменение порядка элементов */
			}
		}
	}
}
\end{lstlisting}

\end{document}
