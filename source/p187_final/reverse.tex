\documentclass{article}

\usepackage[T2A]{fontenc}
\usepackage[utf8]{inputenc}
\usepackage[english, russian]{babel}
\usepackage[left = 2cm]{geometry}
\usepackage{listings}
\usepackage{amssymb}
\usepackage{caption}

\lstset{
  language=C,
  extendedchars=\true,
  keepspaces=true,
  breaklines=true,
  aboveskip=-5pt,
  tabsize=1,
  morekeywords={Вход:, Выход:, true, false, bool, NULL}
}

\begin{document}

\begin{lstlisting}[escapechar=\%]
%\noindent\textbf{Вход:} k - номер элемента, задающий отрезок массива P, подлежащий перестановке в обратном порядке.%
%\textbf{Выход:} первые k элементов массива P переставлены в обратном порядке.%
void Reverse(int k)
{
	int j = 0; /* нижняя граница обращаемого диапазона */
	while(j < k)
	{
		swap(j, k); /* меняем местами элементы */
		j = j + 1; /* увеличиваем нижнюю границу */
		k = k - 1; /* уменьшаем верхнюю границу */
	}
}
\end{lstlisting}

\end{document}
