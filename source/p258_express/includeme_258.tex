\vspace{5pt} \hrule
\begin{lstlisting}[caption={Поиск в ширину и в глубину}, label=p258_graph_search, escapechar=`]
`\noindent\textbf{Вход:} граф $G(V,E)$, представленный списками смежности $\Gamma$.\\`
`\textbf{Выход:} последовательность вершин обхода.`
vector* graph_search(int p, list_node **G) {
  vector *out = vector_create();
  T *t = T_create();
  int x[p];
  for (int i = 0; i < p; i++) {
    x[i] = 0; /* вначале все вершины не отмечены */
  }
  int v = 0; /* начало обхода - произвольная вершина */
  T_push(t, v); /* помещаем `$v$` в структуру данных `$T$`... */
  x[v] = 1; /* ... и отмечаем вершину `$v$` */
  do {
    int u = T_pop(t); /* извлекаем вершину из структуры данных `$T$`... */
    /* ... и возвращаем её в качестве очередной пройденной вершины */
    vector_push_back(out, u);
    list_node *iter = G[u];
    while (iter != NULL) {
      int w = iter->v;
      iter = iter->n;
      if (x[w] == 0) {
        T_push(t, w); /* помещаем `$w$` в структуру данных `$T$`... */
        x[w] = 1; /* ... и отмечаем вершину `$w$` */
      }
    }
  } while (T_is_empty(t) != 0);
  T_destroy(t);
  return out;
}
\end{lstlisting}
